%
% File semeval2020.tex
%
% Nathan Schneider
%% Based on the style files for COLING-2020 (feiliu@cs.ucf.edu & liang.huang.sh@gmail.com), which were, in turn,
%% Based on the style files for COLING-2018, which were, in turn,
%% Based on the style files for COLING-2016, which were, in turn,
%% Based on the style files for COLING-2014, which were, in turn,
%% Based on the style files for ACL-2014, which were, in turn,
%% Based on the style files for ACL-2013, which were, in turn,
%% Based on the style files for ACL-2012, which were, in turn,
%% based on the style files for ACL-2011, which were, in turn,
%% based on the style files for ACL-2010, which were, in turn,
%% based on the style files for ACL-IJCNLP-2009, which were, in turn,
%% based on the style files for EACL-2009 and IJCNLP-2008...

%% Based on the style files for EACL 2006 by
%%e.agirre@ehu.es or Sergi.Balari@uab.es
%% and that of ACL 08 by Joakim Nivre and Noah Smith

\documentclass[11pt]{article}
\usepackage{geometry}
\usepackage{coling2020}
\usepackage{times}
\usepackage{url}
\usepackage{latexsym}
\usepackage{microtype}
\usepackage{booktabs}
\usepackage{graphicx}
\usepackage{courier}
\usepackage{multirow}

\hyphenation{an-aly-sis}
\hyphenation{an-aly-ses}
\hyphenation{Sem-Eval}
%\setlength\titlebox{5cm}
\colingfinalcopy % Uncomment this line for all SemEval submissions

% You can expand the titlebox if you need extra space
% to show all the authors. Please do not make the titlebox
% smaller than 5cm (the original size); we will check this
% in the camera-ready version and ask you to change it back.


\title{T\"uKaPo at SemEval-2020 Task 6: Def(n)tly not BERT:\\
Definition Extraction using pre-BERT Methods in a post-BERT World}

% \author{Madeeswaran Kannan \qquad Haemanth Shanthi Ponnusamy\\
% 	Department of Linguistics \\
% 	University of T\"ubingen, Germany \\
% 	{\tt \{mkannan,???\}@sfs.uni-tuebingen.de} \\}
\author{Madeeswaran Kannan \\
  Department of Linguistics \\
  University of T\"ubingen, Germany \\
  {\tt mkannan@sfs.uni-tuebingen.de} \\\And
  Haemanth Shanthi Ponnusamy \\
  Department of Linguistics \\
  University of T\"ubingen, Germany \\
  {\tt ???@sfs.uni-tuebingen.de} \\}

\date{}

\begin{document}
\setlength{\parindent}{0pt}.

\maketitle
\begin{abstract}
  We describe our system (T\"{u}KaPo) submitted for Task 6: DeftEval, at SemEval 2020.
  We developed and evaluated multiple neural network models based on CNNs and LSTMs to
  perform binary classification of sentences containing definitions. Our final model
  achieved a F1 score of 0.6851 in subtask 1.
\end{abstract}

\renewcommand*{\arraystretch}{1.25}

\section{Introduction}

Definition detection and extraction has been a well-researched topic in NLP research
for over a decade.
%
% The following footnote without marker is needed for the camera-ready
% version of the paper.
% Comment out the instructions (first text) and uncomment the 8 lines
% under "final paper" for your variant of English.
%
\blfootnote{
    %
    % for review submission
    %
    % \hspace{-0.65cm}  % space normally used by the marker
    % Place licence statement here for the camera-ready version. See
    % Section~\ref{licence} of the instructions for preparing a
    % manuscript.
    %
    % % final paper: en-uk version
    %
    % \hspace{-0.65cm}  % space normally used by the marker
    % This work is licensed under a Creative Commons
    % Attribution 4.0 International Licence.
    % Licence details:
    % \url{http://creativecommons.org/licenses/by/4.0/}.
    %
    % % final paper: en-us version
    %
    \hspace{-0.65cm}  % space normally used by the marker
    This work is licensed under a Creative Commons
    Attribution 4.0 International License.
    License details:
    \url{http://creativecommons.org/licenses/by/4.0/}.
}


\section{Background}
The DeftEval shared task is based around the English-language DEFT (Definition Extraction From Texts) corpus \cite{spala-etal-2019-deft}.
It consists of annotated text extracted from the following semi-structured and free-text sources: 2017 SEC contract filings from the US Securities and
Exchange Commission EDGAR database\footnote{https://www.sec.gov/}, and open-source textbooks from OpenStax CNX\footnote{https://cnx.org/}. The
latter encompasses topics from areas of biology, history, physics, psychology, economics, sociology, and government.
Compared to similar existing definition corpora such as WCL \cite{navigli2010learning} and W00 \cite{jin2013mining}, the data offered by the DEFT corpus is
larger in size (23,746 sentences; 11,004 positive annotations) while also providing finer-grained feature annotations
(c.f figure \ref{deft-annotation-scheme}\footnote{Additionally, relationships between terms and definitions are also annotated.}).\\

The shared tasks consists of three subtasks: 1) Sentence Classification (classify if a sentence contains a definition or not),
2) Sequence Labeling (label each token with BIO tags according to the corpus specification), and 3) Relation Classification
(label the relations between each tag according to the corpus specification). We participated in the first subtask.\\

\begin{table}[]
  \centering
  \resizebox{\textwidth}{!}{%
  \begin{tabular}{@{}ll@{}}
  \toprule
  Tag                    & Description                                                                                \\ \midrule
  Term                   & Primary term                                                                               \\
  Alias Term             & Secondary, less common name for the primary term                                           \\
  Ordered Term           & Multiple inseparable terms that have matching sets of definitions                           \\
  Referential Term       & NP reference to a previously mentioned term                                                \\
  Definition             & Primary definition of a term                                                                \\
  Secondary Definition   & Supplemental information crossing a sentence boundary that could be part of the definition \\
  Ordered Definition     & Multiple inseparable definitions that have matching sets of terms                           \\
  Referential Definition & NP reference to a previously mentioned definition                                           \\
  Qualifier              & Specific date, location, or condition under which the definition holds                       \\ \bottomrule
  \end{tabular}%
  }
  \caption{DEFT Tag Schema}
  \label{deft-annotation-scheme}
\end{table}


Training and development data is common for all three subtasks. It is presented in a tab-delimited CONLL-2003-like
 \cite{sang2003introduction} format where each line represents a token and its features:\\

\begin{small}
  \centerline{\texttt{$[TOKEN]\hspace{6pt}[SOURCE]\hspace{6pt}[START\_CHAR]\hspace{6pt}[END\_CHAR]\hspace{6pt}
  [TAG]\hspace{6pt}[TAG\_ID]\hspace{6pt}[ROOT\_ID]\hspace{6pt}[RELATION]$}}
\end{small}
\bigskip

{\small\texttt{SOURCE}} is the source text file, {\small\texttt{START\_CHAR}} and {\small\texttt{END\_CHAR}} are the character index
boundaries of the token, {\small\texttt{TAG}} is the BIO label of the token, {\small\texttt{TAG\_ID}} is the ID
associated with the {\small\texttt{TAG}}, {\small\texttt{ROOT\_ID}} is the ID associated with the root of this relation
(if any), and {\small\texttt{RELATION}} is the relation tag of the token.\\

The test data for the first subtask is presented in the following CONLL-2003-like format:
{\small\texttt{$[SENTENCE]\hspace{6pt}[BIN\_TAG]$}}. {\small\texttt{BIN\_TAG}} is $1$ if
{\small\texttt{SENTENCE}} contains a definition, $0$ otherwise. During the training for the first subtask, the training and
development datasets were converted into the same format was the test dataset using a script provided with the corpus. A positive
label was associated with every sentence that contained tokens with {\small\texttt{B-Definition}} or {\small\texttt{I-Definition}} tags;
all other sentences were associated with a negative label.


\subsection{Related Work}
In the early days of definition extraction, rule-based approaches leveraging  linguistic features showed promise.
\newcite{westerhout2009definition} used a combination of linguistic information (n-grams, syntactic features) and
structural information (position in sentence, layout) to extract definitions from Dutch texts. Such approaches, however,
were found to be dependent on language and domain and scale poorly. Later research incorporated machine learning
methods to encode lexical and syntactic features as word vectors \cite{del2014coping}. \newcite{noraset2017definition}
tackled the problem as a language modelling task over learned definition embeddings. \newcite{espinosa2015definition}
derive feature vectors from entity-linking sources and sense-disambiguated word embeddings. More recently,
\newcite{anke2018syntactically} use convolutional and recurrent neural networks over syntactic dependencies to achieve
state-of-the-art results on the WCL and W00 datasets \cite{navigli2010learning,jin2013mining}.


\section{System Overview}
\subsection{Baseline}
We developed and iterated over both LSTM-based \cite{hochreiter1997long} recurrent and convolutional \cite{o2015introduction}
neural network models. Our baseline RNN architecture is a network of a single bidirectional LSTM layer followed by two feed-forward layers (w/t Dropout) and a final sigmoid-activated read-out layer. This architecture is implemented by model \emph{BL-RNN}
whose input layer accepts sequences of features vectors. Our baseline hybrid-CNN architecture is implemented by model
\emph{BL-CNN} that is based on the work by \newcite{anke2018syntactically}. It accepts feature vector
sequences that are passed through a one-dimensional convolutional filter and a max-pooling layer, followed by a BiLSTM and read-out
layer. The intuition behind combining convolutional and recurrent layers is to leverage the implicit local feature-extraction performed by the convolutional layers to refine the final representation passed to the recurrent layer, which then accounts for global features. The input sequences are composed as concatenations of vectors of individual features at the token level,
resulting in a homogenous representation, e.g. each token is encoded as the concatenation of a $n$-dimensional word vector,
a $m$-dimensional one-hot encoded POS tag vector, etc.\\

We conducted several experiments with the above two architectures and iterated over successful models. Training data was split into 90-10 train-test data splits.
10\% of the train data split was used for validation. All models were trained for 100 epochs with an early-stopping mechanism
that monitored the validation loss over the last 10 epochs. Batch size was set to 128, and ADAM \cite{kingma2014adam} was used
as the binary cross-entropy optimizer. URLs were stripped from token sequences as a preprocessing step. The results of our experiments are listed in table \ref{tab:baseline-experiments}. All experiments were carried out multiple times; the reported
figures were averaged over three iterations.\\

\begin{table}[]
  \centering
  \resizebox{\textwidth}{!}{%
  \begin{tabular}{@{}llclccr@{}}
  \toprule
  Experiment & Model & Word Embeddings & Features & Precision & Recall & F1-Score \\ \midrule
  \begin{tabular}[c]{@{}l@{}}1. W/o Semantic\\     Information\end{tabular} & BL-RNN & - & POS + Deps & 0.56 & 0.75 & 0.64 \\ \midrule
  \multirow{4}{*}{\begin{tabular}[c]{@{}l@{}}2. W/t Semantic\\     Information\end{tabular}} & \multirow{2}{*}{BL-RNN} & Glove & \multirow{4}{*}{Tokens + Deps} & 0.74 & 0.63 & \textbf{0.68} \\
   &  & w2v &  & 0.71 & 0.60 & 0.65 \\
   & \multirow{2}{*}{BL-CNN} & Glove &  & 0.72 & 0.62 & \textbf{0.67} \\
   &  & w2v &  & 0.72 & 0.58 & 0.64 \\ \midrule
  \multirow{8}{*}{\begin{tabular}[c]{@{}l@{}}3. Effect of punctuation \&\\     dependency relations\end{tabular}} & \multirow{4}{*}{BL-RNN} & \multirow{8}{*}{Glove} & Tokens + POS & 0.75 & 0.62 & 0.68 \\
   &  &  & Tokens + Deps + POS & 0.76 & 0.58 & 0.66 \\
   &  &  & Tokens + POS + Punct & 0.76 & 0.64 & \textbf{0.69} \\
   &  &  & Tokens + Deps + POS + Punct & 0.76 & 0.62 & 0.68 \\
   & \multirow{4}{*}{BL-CNN} &  & Tokens + POS & 0.74 & 0.65 & 0.69 \\
   &  &  & Tokens + Deps + POS & 0.77 & 0.67 & \textbf{0.71} \\
   &  &  & Tokens + POS + Punct & 0.77 & 0.64 & 0.70 \\
   &  &  & Tokens + Deps + POS + Punct & 0.79 & 0.62 & 0.70 \\ \midrule
  \multirow{2}{*}{4. Final Model} & \multirow{2}{*}{FINAL-HYBRID} & \multirow{2}{*}{Glove} & Tokens + POS + Punct & 0.77 & 0.64 & 0.70 \\
   &  &  & Tokens + Deps + POS + Punct & 0.75 & 0.70 & \textbf{0.73} \\ \bottomrule
  \end{tabular}%
  }
  \caption{Results of experiments performed on the baseline \& final models.}
  \label{tab:baseline-experiments}
\end{table}

\subsection{Influence of Semantic Information}
Our initial experiments were premised on the hypothesis that neural definition extraction can be modelled on primarily
morphosyntactic features while excluding or restricting the use of semantic and lexical information. By limiting the influence
of semantics, we expected to train a model that generalized well over multiple domains by virtue of being less susceptible to
lexical cues that could potentially act as distractors. To test this, we trained the RNN model on a concatenation of
part-of-speech tag and dependency relation sequences. Similarly, two more RNN models were trained on word embedding and dependency
relation sequences, their word embedding matrix initialized with 300-dimensional pre-trained GloVe \cite{pennington2014glove} and
word2vec \cite{mikolov2013efficient} embeddings respectively\footnote{The GloVe and w2v embeddings were trained on the Common Crawl
and Google News corpora respectively.}. The results proved our hypothesis to be flawed, as the models enriched with semantic
information provided by the word embeddings consistently out-performed our syntax-only model. This result also carried over to the
hybrid-CNN models. Overall, we found that models trained with GloVe embeddings to be more performant than those trained with w2v.\\

\subsection{Feature Modelling}
Building upon the findings of the previous experiments, we tested the effect of combining punctuation and part-of-speech tags. It
was immediately evident that replacing the \emph{PUNCT} POS tag with the punctuation character occurring at that position had a
positive effect on the model's performance. Beyond the implicit increase in information offered by the actual character, it
also reaffirms the importance of syntactic features in this task. The addition of dependency relation features, however, has a less
immediately-obvious impact. The RNN model sees a reduction in performance while the hybrid-CNN model fares better. Upon further
investigation, we determined that the input encoding scheme's attempt to homogenize feature vectors across disparate features, viz.,
combining sequential (token-level) features (token, POS) with non-sequential (sentence-level) features (dependencies), which
hindered the recurrent model from optimally exploiting the former. With this key insight, we were able to rearchitect our model
to learn a representation that composes both token and sentence-level features in an efficient way.

\subsection{Final Architecture}
Our final architecture is informed by the findings of our previous experiments. It accepts five inputs: At the token-level, both  token
and part-of-speech tags (w/t punctuation) are used. Pre-trained GloVe embeddings are used for tokens, while embeddings for POS tags
are learned on-the-fly. The concatenation of both embeddings is passed through two "feature extraction" units that consist of a
BiLSTM (to target sequence/global information), and a 1D-Conv + MaxPool layer (to target local information). At sentence-level,
dependency information is encoded as the concatenation of the embeddings of the head, modifier and dependency label of each relation.
This is connected to two stripped-down "feature extraction" units without the BiLSTM layer, since dependency relations are
sequentially independent. Finally, the extracted representations of both token- and sentence-level features are concatenated and
connected to a feed-forward layer and a read-out layer.\\

The above architecture's separation of feature-extraction at token and sentential levels allows their information to be combined
at a higher level in the network. And we indeed see a marked improvement when this model is trained with dependency information.
The model achieved a best F1-score of $0.76$ during development.


\begin{table}[]
  \centering
  \resizebox{3.5in}{!}{%
  \begin{tabular}{@{}llcc@{}}
  \toprule
  \multicolumn{2}{c}{\multirow{2}{*}{Hyperparameter}} & \multicolumn{2}{c}{Value} \\
  \multicolumn{2}{c}{} & Token-Level & Sentence-Level \\ \midrule
  \multirow{3}{*}{\begin{tabular}[c]{@{}l@{}}Embeddings\\ Dim\end{tabular}} & Word & \multicolumn{2}{c}{300} \\
   & POS & \multicolumn{2}{c}{32} \\
   & Dep. Label & \multicolumn{2}{c}{32} \\
  \multirow{4}{*}{\begin{tabular}[c]{@{}l@{}}Feature Extractor \\ Units\\ (Unit 1, Unit 2)\end{tabular}} & LSTM Units & 128, 64 & - \\
   & Conv. Filters & 128, 64 & 64, 32 \\
   & Conv. Kernel Size & \multicolumn{2}{c}{3, 3} \\
   & MaxPooling Pool Size & \multicolumn{2}{c}{2, 2} \\
  \multicolumn{2}{l}{L2 Regularizer Beta} & \multicolumn{2}{c}{0.001} \\
  \multicolumn{2}{l}{Feed-forward Layer Units} & \multicolumn{2}{c}{24} \\ \bottomrule
  \end{tabular}%
  }
  \caption{Hyperparameters for the final model}
  \label{tab:final-hyperparameters}
\end{table}


\section{Evaluation \& Results}



\section{Conclusion}



% include your own bib file like this:
\bibliographystyle{coling}
\bibliography{semeval2020}



\end{document}
